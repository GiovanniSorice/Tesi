%!TEX root = ../dissertation.tex

\chapter{Evoluzione dello stage}\label{ch:evoluzione-dello-stage}
In questo capitolo verrà introdotto il progetto, spiegandone vincoli, obiettivi, pianificazione e motivi della scelta.
\section{La proposta di stage}
La proposta di stage mi è stata fatta nei mesi successivi alla visita all'azienda svolta attraverso l'Università di Padova. Fin da subito il complesso e lo spirito aziendale mi sono sembrati adatti ad ospitare degli studenti volenterosi di sperimentare nuove tecnologie.
\\
Durante i colloqui con il Dott. Michele Giusto, mi è stato esposto un progetto incentrato sui \emph{big data} e il rilevamento delle anomalie che corrispondeva al mio bisogno di affrontare nuove sfide e approfondire argomenti non trattati durante i corsi universitari.
\\
Negli ultimi anni infatti, tra le tecnologie che hanno riscosso più successo e fama, troviamo i \emph{big data} e tecniche di intelligenza artificiale con particolare attenzione al \emph{machine learning}.
Per questo motivo e per il mio già precedente interessamento verso queste tecnologie, ho deciso di dare forma agli obiettivi di questo progetto. 

\section{Il progetto}
Il progetto si prefissava l'obiettivo di analizzare e costruire l'intero ciclo di vita del dato, secondo metodi, strumenti e tecniche innovative.
\\
Per essere maggiormente concreti, il progetto prevedeva l'implementazione di un sistema di gestione dei dati aziendali del cliente, che quindi comportasse l'estrapolazione di informazioni dal dato grezzo, con successiva classificazione degli utenti che interagivano con il sistema stesso, in modo da intercettare eventuali comportamenti sospetti.
\\
Per svolgere questo progetto, prima del mio arrivo e dell'effettiva proposta, l'azienda ospitante ha svolto uno studio di fattibilità in modo da sottolineare eventuali carenze di informazioni, criticità e punti a favore del progetto.
\\
Durante il mio periodo di permanenza, il progetto è stato portato al termine seguendo le indicazioni iniziali e con alcuni aggiustamenti durante il percorso. Inoltre, sono state rispettate le fondamentali indicazioni del tutor aziendale che mi ha affiancato durante lo stage.
\\
Terminato il mio stage il prodotto risultava conforme alle aspettative e facilmente manutenibile grazie alla documentazione redatta durante i periodi di studio del problema, sviluppo e test.
\subsection{Obiettivi}
Durante i colloqui di esposizione del progetto di stage, sono stati definiti alcuni obiettivi da soddisfare al termine dello stesso.\\
Essi si dividono in:
\begin{itemize}
	\item Tutti gli obiettivi che sono fondamentali per la corretta riuscita del progetto, anche detti, obiettivi obbligatori:
	\begin{itemize}
		\item Il sistema deve essere in grado di aggiornare periodicamente la concezione di normalità, per adattarsi al contesto dinamico;	 
		\item Il sistema deve essere in grado di ricevere i flussi \emph{batch} ed archiviarli dove desiderato;
		\item Il sistema deve essere in grado di processare i flussi \emph{batch} producendo i \emph{dataset} di output richiesti;
		\item I processi devono essere programmati per avviarsi periodicamente.
	\end{itemize}
	\item Obiettivi desiderabili, quindi tutti gli obiettivi che porterebbero ad una maggiore completezza del progetto di stage ma che non sono fondamentali per la sua buona riuscita:
		\begin{itemize}
		 \item Il sistema deve riuscire ad aggiornarsi continuamente senza interrompere l'erogazione del servizio;
		 \item Il sistema deve essere in grado di scrivere i risultati su una struttura interrogabile via SQL;
		 \item Il sistema deve poter scalare le risorse in base al volume di dati.
		\end{itemize}	
	\item Tutti gli obiettivi che sono marginali al progetto ma che porterebbero comunque valore aggiunto allo stesso, anche detti, obiettivi facoltativi:
	\begin{itemize}
		\item Il sistema deve poter lavorare su dati archiviati su Elasticsearch;
		\item Il sistema deve poter essere in grado di scrivere i propri output su \emph{database NoSQL}. 
	\end{itemize}
\end{itemize}
\subsection{Vincoli}
Durante i colloqui di esposizione del progetto di stage, sono sorti alcuni vincoli da soddisfare per svolgerlo al meglio.
\subsubsection{Vincoli temporali}
La durata dello stage doveva essere al minimo di 300 ore e al massimo di 320, in modo da rispettare i crediti stabiliti all'interno del piano di studi, svolte tutte presso la sede in via Robert Koch 1/4 a Milano negli orari 9.00-13.00, 14.00-18.00.
\subsubsection{Vincoli tecnologici}
Utilizzo della suite Google Cloud per lo sviluppo dell'\emph{ingestion} dei dati e di Spark MLlib per la parte di riconoscimento delle anomalie.
\section{Motivi della scelta}
La scelta di abbracciare il progetto di Data Reply è dovuto principalmente a 4 motivi:
\begin{itemize}
	\item Proposta di un progetto interessante con esperti del settore;
	\item Azienda dinamica e giovane con grandi ambizioni;
	\item Argomenti e tecnologie molto ricercate nell'ultimo periodo che mi permetteranno di inserirmi al meglio nel mondo del lavoro;
	\item Possibilità di svolgere lo stage presso la sede di Milano, città molto dinamica e innovativa.
\end{itemize}
Questi motivi mi hanno convinto a scegliere questa proposta per mettere alla prova le mie abilità.

\section{Pianificazione del lavoro}
Durante i colloqui svolti con il tutor aziendale, è stato redatto il piano di lavoro. Ciò ha portato la suddivisione dello stage in 8 parti, ognuna della durata di una settimana.
\begin{itemize}
	\item[] \textbf{Prima Settimana (40 ore)}
	\begin{itemize}
		\item Incontro con persone coinvolte nel progetto per discutere i requisiti e le richieste
		relativamente al sistema da sviluppare;
		\item Verifica credenziali e strumenti di lavoro assegnati;
		\item Presa visione dell’infrastruttura esistente;
		\item Formazione sulle tecnologie adottate.
	\end{itemize}
	\item[] \textbf{Seconda Settimana (40 ore)} 
	\begin{itemize}
		\item Comprensione strumenti di storicizzazione;
		\item Analisi funzionamento \emph{filesystem} distribuito;
		\item Produzione script Python per produzione dati simulati.
	\end{itemize}
	\item[] \textbf{Terza Settimana (40 ore)} 
	\begin{itemize}
		\item Caricamento dati su \emph{filesystem} distribuito;
		\item Securizzazione accesso ai dati;
		\item Studio teorico approccio statistico al problema.
	\end{itemize}
	\item[] \textbf{Quarta Settimana (40 ore)} 
	\begin{itemize}
		\item Comprensione strumenti di processamento;
		\item Analisi funzionamento \emph{RDD};
		\item Implementazione processo \emph{batch} con Python e Spark.        
	\end{itemize}
	\item[] \textbf{Quinta Settimana (40 ore)} 
	\begin{itemize}
		\item Analisi funzionamento \emph{Dataframe};
		\item Implementazione processo di elaborazione Spark;
		\item Ingegnerizzazione soluzione \emph{batch}.
	\end{itemize}
	\item[] \textbf{Sesta Settimana (40 ore)} 
	\begin{itemize}
		\item Test prestazionale processo Spark;
		\item Tuning prestazionale;
		\item Studio applicazione real-time.
	\end{itemize}
	\item[] \textbf{Settima Settimana (40 ore)} 
	\begin{itemize}
		\item Storicizzazione dati su \emph{storage} persistente;
		\item Studio funzionamento Kafka;
		\item Implementazione processo real-time con Spark Streaming.
	\end{itemize}
	\item[] \textbf{Ottava Settimana - Conclusione (40 ore)} 
	\begin{itemize}
		\item Costruzione \emph{layer} accesso al dato persistente;
		\item Ingegnerizzazione soluzione Spark Streaming;
		\item Integrazione \emph{storage} persistente.
	\end{itemize}
\end{itemize}
