%!TEX root = ../dissertation.tex
\begin{savequote}[75mm]
Nulla facilisi. In vel sem. Morbi id urna in diam dignissim feugiat. Proin molestie tortor eu velit. Aliquam erat volutpat. Nullam ultrices, diam tempus vulputate egestas, eros pede varius leo.
\qauthor{Quoteauthor Lastname}
\end{savequote}

\chapter{Evoluzione dello stage}
Perchè?
\section{La proposta di stage}
La proposta di stage mi è stata fatta nei mesi successivi alla visita all'azienda svolta attraverso l'Università di Padova. Fin da subito il complesso e lo spirito aziendale mi sono sembrati adatti ad ospitare degli studenti volenterosi di sperimentare nuove tecnologie. 
Negli ultimi anni, tra le tecnologie che hanno riscosso più successo e fama, troviamo i Big Data e tecniche di intelligenza artificiale con particolare attenzione al machine learning. 




\section{Il progetto}
Il progetto si prefissava l'obiettivo di analizzare e costruire l'intero ciclo di vita del dato, secondo metodi, strumenti e tecniche innovative e decise dal cliente.
Quindi, per essere maggiormente concreti, il progetto prevedeva l'implementazione di un sistema di \gls{ingestion} dei dati aziendali del cliente con successiva classificazione degli utenti che interagivano con il sistema stesso, in modo da intercettare eventuali comportamenti sospetti.
\\
Per svolgere questo progetto, prima del mio arrivo e dell'effettiva proposta, l'azienda ospitante ha svolto uno studio di fattibilità in modo da sottolineare eventuali carenze di informazioni, criticità e punti a favore del progetto.
Durante il mio periodo di permanenza, il progetto è stato portato al termine seguendo le indicazioni iniziali e con alcuni aggiustamenti durante il percorso. Inoltre, sono state rispettate le fondamentali indicazioni del tutor aziendale che ha affiancato durante lo stage.
Terminato il mio stage il prodotto risultava conforme alle aspettative e facilmente manutenibile grazie alla documentazione redatta durante il periodo di studio del problema, sviluppo e test.
Spiegarlo ad ampio raggio (prima di me, con me e dopo di me)
\subsection{Obiettivi}
Durante i colloqui di esposizione del progetto di stage, sono sorti alcuni obiettivi da soddisfare al termine dello stesso.\\
Essi si dividono in:
\begin{itemize}
	\item Tutti quegli obiettivi che sono fondamentali per la corretta riuscita del progetto, anche detti, obiettivi obbligatori:
	\begin{itemize}
		\item 
	\end{itemize}
	\item Obiettivi desiderabili, cioè, quegli obiettivi che porterebbero ad una maggiore completezza del progetto di stage ma che non sono fondamentali per la sua buona riuscita:
		\begin{itemize}
		\item 
		\end{itemize}	
	\item Tutti quegli obiettivi che sono marginali al progetto ma che porterebbero comunque valore aggiunto allo stesso, anche detti, obiettivi facoltativi,:
	\begin{itemize}
		\item 
	\end{itemize}
\end{itemize}

\subsection{Vincoli}
\section{Motivi della scelta}
Perchè Reply?