%!TEX root = ../dissertation.tex
\begin{savequote}[75mm]
Nulla facilisi. In vel sem. Morbi id urna in diam dignissim feugiat. Proin molestie tortor eu velit. Aliquam erat volutpat. Nullam ultrices, diam tempus vulputate egestas, eros pede varius leo.
\qauthor{Quoteauthor Lastname}
\end{savequote}

\chapter{Sviluppo}
\section{Tecnologie}
In questa sezione, verranno elencate e introdotte le tecnologie che sono state utilizzate nel progetto di stage.
\subsection{Google Cloud}
Il principale servizio utilizzato durante il progetto è stato Google Cloud, in esso troviamo una moltitudine di strumenti utili allo sviluppo, all'esecuzione e all'interazione di progetti riguardanti Big Data e Intelligenza Artificiale. Inoltre, importante risulta la possibilità di consultare, manipolare e presentare i dati.
Il progetto Google Cloud (anche detto Google Cloude Platform), viene lanciato nel 2008 con uno dei suoi servizi di punta ancora oggi \gls{App Engine}. Nell'arco degli anni sia il bacino di utenti, sia i servizi offerti sono via via aumentati. Oggigiorno i suoi \href{https://cloud.google.com/customers/#/}{clienti} sono molti, tra cui grandi aziende, e le funzionalità di spicco sono molteplici, i campi di maggiore interesse riguardano strumenti di Compute, Storage \& Databases, Big Data, Cloud AI e IoT.
Il resto della sezione introduce il prodotti e i servizi GCP utilizzati durante lo stage.	
\subsubsection{Google Storage}

\href{https://cloud.google.com/storage/}
\subsubsection{BigQuery}
\url{https://cloud.google.com/bigquery/}
\subsubsection{Dataflow}
\url{https://cloud.google.com/customers/#/}
\subsubsection{Storage}
\url{https://cloud.google.com/storage/}
\subsubsection{Data Studio}
\url{https://support.google.com/datastudio/answer/6390659?hl=it}
\subsection{Apache NiFi}
\url{https://nifi.apache.org/docs/nifi-docs/html/getting-started.html}
\subsection{Apache Beam}
\url{https://beam.apache.org/documentation/programming-guide/}
\subsection{AirFlow}
\url{https://airflow.apache.org/}\\
\section{Integrazione}
Come?

