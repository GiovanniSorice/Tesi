%!TEX root = ../dissertation.tex
\chapter{Introduzione}
\label{Introduzione}
Dove?
parte introduttiva, spiegazione cosa leggeremo
In questo capitolo verrà introdotta l'azienda ospitante e le tecnologie da essa utilizzatel.
\section{L'azieda}
Data Reply S.r.l fa parte delle aziende del gruppo Reply S.p.a e si occupa del mondo Big Data, Data Science e Artificial Intelligence. L'azienda è giovane sia nella sua creazione, dato che è stata fondata nel 2013, sia dal punto di vista della composizione del personale. Questo le permette di avere flessibilità e freschezza mentale (in termini di idee) ma allo stesso tempo know how e conoscenza del settore messa in campo dagli elementi con maggiore esperienza.

\begin{figure}
	\centering
	\includegraphics[scale=2]{figures/data-reply-logo}
	\caption[Short figure name.]{Logo Data Reply S.r.l.
		\label{fig:logoDataReply}}
\end{figure}

Il mercato di Data Reply è quello della consulenza, un mondo spesso ostico e molto complesso ma che ha visto un importante crescita degli investimenti in Italia negli ultimi anni.
\subsection{Le tecnologie da esplorare}
Le principali tecnologie utilizzate dall'azienda riguardano i settori dei Big Data, Data Science e Artificial Intelligence. Troviamo un importante utilizzo delle suite di prodotti cloud, come \gls{AWS} e \gls{Google Cloud}, e di prodotti da poter utilizzare all'interno di macchine proprie, come \gls{Cloudera}. Infine, anche l'utilizzo di framework per il calcolo distribuito, come \gls{Apache Spark}, sono di fondamentale importanza.
\section{Processi del progetto}
\subsection{Metodologia di sviluppo}
All'interno di Data Reply vengono utilizzati diverse metodologie di sviluppo, che vanno dalle metodologie agile, nella quale vediamo tra le più utilizzate \gls{Scrum} e \gls{Kanban}, a metodologie a cascata. Essendo un'azienda di consulenza, spesso si trova a lavorare anche con altre aziende allo stesso progetto e per questo motivo non è raro trovare metodologie più rigide ai cambiamenti come le metodologie a cascata.
\subsection{Strumenti di supporto}
Gli strumenti di supporto utilizzati, come le metodologie di sviluppo,  sono decisi in base al progetto, al cliente e in accordo con le altre aziende che lavoreranno al progetto stesso.
Di consueto, l'azienda propone l'utilizzo di \gls{GitLab} come strumento di versionamento e \gls{GitLab Mattermost} per le comunicazioni informali tra i componenti del progetto e dell'azienda.
\subsection{Propensione alla modernizzazione}
L'azienda cerca sempre di rinnovarsi e rimanere al passo con le ultime tecnologie in modo da poter proporre ai propri clienti il giusto compromesso tra novità e affidabilità. Spesso infatti, vi sono progetti mirati allo studio ed utilizzo di tecnologie e prodotti da poco sul mercato, così da garantire un vantaggio competitivo sulle altre aziende.
Questo si può anche notare dai raduni svolti dall'azienda madre Reply nella quale Data Reply cerca sempre di portare progetti innovativi e che vadano a toccare le ultime tendenze del momento.
